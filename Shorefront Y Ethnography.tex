\documentclass[twoside]{article} \usepackage[pdftex]{graphicx}
\usepackage{fullpage}
\usepackage[natbibapa]{apacite}
\usepackage{indentfirst}
\usepackage{wrapfig}
\usepackage{varwidth}
\usepackage{multicol}
\pagenumbering{gobble}
\begin{document}

\title{The Shorefront YM-YWHA of Brighton - Manhattan Beach:\\A Nodal
Point in an American Immigrant Community}
\date{\today}
\author{Aleksey Bilogur\\
\includegraphics[width=400pt]{ShorefrontLogo.jpg}
}
\maketitle

\newpage
\begin{multicols}{2}

\noindent
\setlength\fboxsep{0pt}
\setlength\fboxrule{0.5pt}
\noindent
\fbox{\includegraphics[width=220pt]{BrightonSign.jpg}}
\vspace{12pt}


An old joke holds that Brighton Beach is conveniently located---near the United
States. Heavy waves of immigration ongoing since before the fall of the Soviet
Union have transformed ``Little Odessa'' into a heavily Russian, overwhelmingly
Jewish, incredibly compacted ethnic enclave. On its commercial streets forests
of closely-packed stores hawk their wares in glaringly colored Cyrillic signage,
shoppers browsing unperturbed by the elevated trains roaring by above; on its
residential avenues aging pre-war apartment blocks stand proud, their fronts
barricaded by elders enjoying the latest Russian literature in the fresh air; on
its boardwalks impressive ethnic reasturants stand shuttered through the day,
only to come violently to life at night, their celebrations often spilling onto
the beach below. For an outside observer to step into the neighborhood is as if
to travel to another country, one where aging Soviet citizens still recognisant
of Stalin's times share space with the cast of Jersey Shore lite ``Russian
Dolls'', English is often a second tongue, and the winning councilman sees fit
to declare himself ``the only frum [religious] candidate'' in his campaign flyers.

\vspace{12pt}
\noindent
\fbox{\includegraphics[width=225pt]{ShorefrontEntrance.jpg}}
\vfill
\columnbreak

\noindent
\fbox{\includegraphics[width=225pt]{ConeyIslandMap.png}}
\vspace{6pt}

This is the community that the Shorefront Y serves. Located in the sleepy
underbelly of the neighborhood, with residential co-ops on one side and the sea
on the other, the center (often simply referred to as �the Jewish Center� by
true locals) is the neighborhood's largest community center and the local anchor
of the transnational Jewish Community Center (JCC) Association. The Shorefront
Y's widely splayed two-story residence is purpose-built for the task, sporting
such amenities as a backyard sports area, gyms, lunchrooms, large amounts of
classroom and office space, a lunchroom, a rooftop playground, and a pool.
Beyond a colorful mural in the small entry lobby the hallways of the center
proper are mostly undecorated, bearing instead copious billboards and notices
inviting the reader to explore the center's cornucopia of community initiatives,
ponder the activities of American Jews, or revel in the achievements and artwork
of the center's day campers and afterschoolers. Here or there signs of the
center's age can be found: a hallway telephone booth with all of its wires long
since cut out; a plaque commemorating the generous contributions of a
philanthropist who bankrolled the center's switch from wooden to tile flooring
in the 1980s.

To locals like Elon the Shorefront Y is not just a place to go to get help when
in need but also binding element of his community. Now a shopkeeper on Brighton
Beach Avenue, the neighborhood's main thoroughfare, Elon is an American citizen
and a fully proficient English speaker---though most of his business occurs in
Russian anyway. He attributes much of his success in this ``new life'' to the
classes that he took on American English at the Y and the financial support he
received from the community on his arrival; he retains relatively expensive
family membership at the center in part as a token of gratitude. Every few
weekends Elon meets up with his friends---many if not most of them formerly
students in the same programs---and the group reminisces about old times and
discusses the new at the beach, the boardwalk restaurants, the nearby billiards
place (``Boardwalk Billiards''), or the center's swimming pool. Some of the most
important Jewish holidays and most interesting cultural events find Elon lining
up at the door with his family (his wife, Gena, and the kids and \emph{their}
kids, when he can get them) and friends to partake in the festivities.

\vspace{6pt}
\noindent
\fbox{\includegraphics[width=225pt]{ShorefrontESLClass.jpg}}
\vspace{6pt}

Jews in the Soviet Union were strictly contained, marked ``J'' for Jew on their
passports, placed under a strict university quota, and subjected to widespread
petty discrimination---so much as approaching a jeweler and asking for a Star of
David could get you thrown in jail. Elon was part of a long line of reasonably
well-off immigrants fleeing such conditions to the prosperity of �the West�,
so-called �refuskis� awarded preferred refugee status under American immigration
laws for reason of religious prosecution. Tens of thousands of Jews left ``the
old country'' for the new every year (the United States maintained a majority
lead over Israel); most arrived by way of New York City and about a third chose
to stay here, usually resettling near American peers in Brighton Beach and other
enclavic neighborhoods in southern Brooklyn. By the time Elon arrived in the
United States in 1994 these migrants were met by what may be one of the most
robust immigration support networks in the United States, run by an alphabet
soup of Jewish interest organizations (NYANA, HIAS, JCC, JASA) with long reach.
In his case it found him right out of the plane---a NYANA (New York Association
for New Americans) volunteer, schedule-book in hand. Two days later Elon was
already sitting in the office of one of the organization's caseworkers,
receiving basic identification, a modest three-month stipend, a few articles of
furniture, housing recommendations, and a letter of introduction for one of the
nearby Jewish center's new immigrant English tutoring programs---Shorefront's. A
month and a half saw him enrollment in the program, a permanent place of
residence, and a job at one of the neighborhood's Russian-language bookstores;
the support and services that NYANA and the Shorefront Y offered saw Elon very
quickly ascend to a modest, but comfortable, new life.

Elon got the job with the help of his English-language tutor, a woman that he
says helped him immensely with adapting to American culture and to the American
way of life. Anastasia Milek, a present day English language tutor at the
Shorefront Y, takes great pride in this aspect of her job; she likens the role
her lessons play to that of the Israeli \emph{ulpanim}, Hebrew language
instructionals in Israel that were set up to handle the diverse voices of the
Jewish diaspora in the \emph{aliyah} (return). As the first regular point of
contact for many of the center�s newly arrived immigrants Milek is similarly
more than just a language tutor---in fact, she goes as far as to say that
calling the class something as mundane as ``Basic English'' is na�ve. Milek aims
to instill in her students an understanding of not just the English language but
of American culture and the American way of life as well, and her classes often
go into tangents on the intricacies of American culture, into discussions on
Texas, pizza, and Uncle Sam. She makes heavy use of audiovisual aids, often
bringing to class examples of the words to be discussed that day---kitchen
utensils, measuring cups, basketballs and spatulas---and often introduces her
students to the concepts they are learning with classic American songs like the
Beatles' ``Can't Buy Me Love''. Those that pass these classes often move on to
more intensive cultural ones meant to prepare the immigrants for taking the
citizenship test-�-but this is in her words a challenge for a secure future, and
working cultural knowledge is necessary to ``thrive today''.

\noindent
\setlength\fboxsep{0pt}
\setlength\fboxrule{0.5pt}
\fbox{\includegraphics[width=220pt]{ShorefrontESL.jpg}}

\end{multicols}
%\noindent
%\begin{center}
%\setlength\fboxsep{1pt}
%\setlength\fboxrule{0.5pt}
%\fbox{\includegraphics[width=450pt]{ChaimDeustch.png}}
%\end{center}

\vspace{20pt}
\begin{multicols}{2}
\noindent
\setlength\fboxsep{0.5pt}
\setlength\fboxrule{0.5pt}
\fbox{\includegraphics[width=222pt]{M&IInternationalFoods.png}}

The neighborhood, however, is changing. Ask a Shorefront Y member what they
think about today's immigrants and you'll likely get some angry fist clenching
over the ``riff raff'' settling in as of late. Jewish immigration to Brighton
Beach reached its peak in the late 90s, and as these immigrants have found work
and status they have begun to move out in a steady stream of out-migration. It
is unfair, as some residents want to claim, to say that the community is dying;
Jews still dominate the neighborhood, as evidenced by the recent district city
council election---�all four candidates Jewish, all but the winning one
ethnically Russian. But as the disgruntlement of the ``old-timers'' shows, the
neighborhood is maturing. Today�s new residents are often either non-Jewish
Ukranians or Pakistanis moving in from the direction of Nostrand Avenue, and
though the co op residences immediately in the vicinity of Shorefront are still
almost wholly dominated by their long term elderly Jewish residents the streets
and residential housing closer to Brighton Beach Avenue, the commercial heart of
the neighborhood, is showing signs of increased diversity. To young,
upwardly mobile children of immigrants Brighton Beach is not a destination in of
itself, but merely a stop on the journey, a quaint, timely place which one would
want to visit occasionally to sample the stores, walk down the boardwalk, or
visit their elders---but no one wants to \emph{live} here. Younger people are
moving out, preferring trendier neighborhoods and residences like Manhattan
Beach�s single family housing: to quote a Midwood (former) classmate of mine,
``Brighton Beach is, frankly, disgusting.'' With the general slowdown in Jewish
immigration they are not being replaced. Though those that may leave often do
may of the neighborhood�s older timers are more reluctant to go; its most
elderly members, living in frozen apartment rentals, often cannot. Both Elon and
my brother's piano instructor, who recently moved out of the neighborhood,
acknowledge that life in Brighton Beach might be boring for ``youngsters'': but
they fret about the community�s dilution and are especially nervous about a
return to the bad old days of the 70s and 80s.

The workers of the Shorefront Y itself are too polite (or, depending on your
perspective, too politically correct) to much acknowledge the shifting
demographic winds, at least publicly, and at a cursory glance it don�t seem to
have affected the center at all. The Shorefront Y is a Jewish community center
after all---a fact proudly displayed on the, well, everything---and though
anyone and everyone is \emph{encouraged} to receive services from the center, in
reality the center�s assistance base reaches few of the new immigrants. The
Shorefront Y is an ethnic enclave within an ethnic enclave; if you don�t blend
in, you�ll likely get more than a few asinine glances from the more typical
constituents relaxing in the center�s lounges or in line for government
paperwork consultation, and possibly even difficulties at the security desk. As
a result most of the immigrants today that receive support from the center do so
remotely, referring to it for one-time assistance with paperwork and durable
essentials like mattresses, hurrying to and from their appointments and not
sampling any of the center's other programs. Elements of soft power are clearly
on display at Shorefront Y, and although it does run some programs specifically
targeting an outside demographic, like an ``All Stars'' afterschool program at
the Kingsborough-affiliated PS 225, they seem cursory compared to the center�s
intracommunity activity.

On the other hand, the community that is present has less control over
Shorefront than many of its members would like. The Shorefront Y is run from the
top down: it is a cog in the intranational machine of the JCC, and so important
positional vacancies and budget breakdowns are filled by a regional board with
no direct relationship to the community center on the ground. As a result,
professional appointments based on resumes and interviews, not on community
activism and ground-level skills; being a Russian language speaker is not even
strictly necessary. In an odd juxtaposition that is not mentioned anywhere on
the center�s website, the JCC even makes JASA (the Jewish Association for the
Aging), which runs the building�s old-age wing and owns an elder-care apartment
block behind the center, pay a fairly sizable rent on its wing of the building.
Rules are maintained on many things---like on-location interviews---that many of
the site�s actual, boots-on-the-ground workers aren't even aware of.

To what extent is the Shorefront Y�s monocular focus acceptable? As part of the
JCC Shorefront is after all a Jewish center, funded by donations from Jewish
patrons (both wealthy and not) interested in Jewish constituents and Jewish
activities. Diluting the overall mission of the center steps on the toes of the
center�s constituents and donators alike, and risks the organization going the
way of NYANA. Despite its ``New York'' name, NYANA was an organization run and
funded by and targeted at Jewish interests (my family actually received our
first apartment and most of our early furniture from the organization; according
to my mom our caseworker even gave her all sorts of strange looks for not
actually being ``fully'' Jewish). When Soviet immigration began to slow down the
organization also began to handle more general immigration---but this caused the
organization�s funding base to slowly evaporate, and after chronic budget
problems it was finally forced shut in 2008. The JCC is a national organization
vastly bigger the exclusively New York City based NYANA, but it still has the
same, probably well-placed fears. One particularly cynical way of looking at
Shorefront Y�s curriculum vitae is that it is merely designed to do just enough
for the general community to not get in trouble for not doing anything at all.

On the offensive front the Shorefront is making a quiet but serious effort to
shift its approach to match the changes in its demographic. For one thing,
immigration services, once the center of the Y, has been shuffled far, far aside
in terms of size and funding---there simply aren't enough Elons entering the
neighborhood anymore. Community services and cultural programs on the other hand
have increased in prominence, and the center now runs many paygated endeavors---
summer camps, day camps, afterschool programs, and a gym---which obviously
target a much wealthier demographic than the penniless immigrant. Even entering
the center has become a process: the Shorefront Y has installed a card swiper at
the entrance, and the general public can---for a low flat rate of three to five
hundred dollars a year---buy a recurring monthly membership that earns them
access to the center�s leisure activity centers, like the tennis and basketball
courts in the backyard sports area and the gym in the basement.

Shorefront Y�s strategy can thus be summarized in one word: cross-marketing.
Take Tamara for instance. Like Elon, Tamara's family immigrated to the United
States in 1994; with only a music degree in hand she was able to use her
parent's financial support to put herself through college. Today she is a
financial professional and her husband works in engineering; together they make
in the vicinity of \$200,000 a year. In the summers they drive their daughter to
the Shorefront Y�s summer day camp, what she explains she feels is the best
option for socializing her in the summer short of taking time off from work and
renting a bungalow upstate (which they have also done in the past). Tamara and
her husband are both extremely busy with their jobs and cannot easily leave
their daughter at home in the summers, and to them the service that Shorefront
Camp provides---a place to drop of their daughter during their busy
workdays---is essential. On closer inspection the camp begins to look like a
(paid) service for families like Tamara's, successful members of the Jewish
middle class with nowhere to keep their children in the summers. The children
are in turn exposed to the center�s other cultural activities, invited to attend
the center�s especially children-themed festivities during holidays and at
cultural events; their parents are in turn exposed to the center�s afterschool,
pre-school activities, and cultural events de jour.

\vfill
%\vspace{6pt}
\noindent
\setlength\fboxsep{0pt}
\setlength\fboxrule{0.5pt}
\fbox{\includegraphics[width=220pt]{ShrorefrontKids.jpg}}
%\vspace{6pt}
\vfill

Another, similar draw is the Lenny Krayzelburg Swim Academy. Part of a ``chain''
of such academies overseen by the four-medal Olympic swimmer---the original is
in a JCC in his hometown, Los Angeles---the academy offers swim lessons for all
ages, including toddlers, and maintains a fairly well-lauded swim team if their
awards cabinet is anything to go by. Well run and exceedingly well marketed, the
swim institute is also a paid service---the students are again wealthier
children from outside the neighborhood. The pool further maintains hours for
members' use: being a fairly large well-maintained indoor pool with a sauna,
this is a popular feature in those colder months when the beach becomes
nonviable, and a big part of the center's membership draw.

\vspace{6pt}
\noindent
\setlength\fboxsep{0.5pt}
\setlength\fboxrule{0.5pt}
\fbox{\includegraphics[width=220pt]{ShorefrontPoolStar.jpg}}
\vspace{6pt}

These two programs are not alone amongst the center's extra-communal
repertoire---but they do represent among the most popular and well-funded (and,
likely, lucrative) programs the center has to offer. Take the Q or the B train
out as far out as Church Avenue and get off at any stop along the way and you
will likely find a board or a poster somewhere in the station advertising one of
the two (and often multiple, and often both); Shorefront Y promos are commonly
found amongst the local area adverts of Russian-language channels like RTVi and
News One. They represent for the center a transitional model, a movement away
from a role as a vital nodal point on the immigrant journey and local area link
in an immigrant community and towards a new, cosmopolitan vision. The Shorefront
Y of the future, at least the one it seems to aim to be, is a gathering point
for the now-successful immigrants and children of immigrants that are drifting
away from their migrant roots, into neighborhoods---and paychecks---far removed
from those of their past. In the meantime it exists in both worlds, offering a
large variety of individually small programs to a wide---but always
Jewish---audience. Thought the future of the neighborhood's ``Jewishness'' has
been cast in doubt, even as the community drifts away the Shorefront Y's
strategic pivot may well secure it a bright, but very different, future.

\end{multicols}

%\newpage
%\nocite{*}
%\bibliographystyle{apacite}
%\bibliography{mybib}
\end{document}